\documentclass{article}
\usepackage[utf8]{inputenc}
\usepackage {mathtools, graphicx, amsfonts, amssymb, comment}
\usepackage{enumerate}
\usepackage{geometry}

\geometry{margin=1.25in}

% http://tex.stackexchange.com/questions/65849/confusion-onehalfspacing-vs-spacing-vs-word-vs-the-world
\linespread{1.25}

\title{Math 414 Final Project}
\author{Matt Gaikema \\ Will}
\date{April 2016}

\begin{document}

\maketitle

%%%%%%%%%%%%%%%%
% INTRODUCTION %
%%%%%%%%%%%%%%%%
\section{Introduction}

In order to efficiently send and receive images, it is often useful to compress them.
One of the many ways of accomplishing this is through wavelets.

There are two types of image compression: lossy and lossless.
\textbf{Lossless} compression means that every bit of information is recovered from the compressed data,
while \textbf{lossy} compression occurs when redundant information is eliminated.
The GIF is an example of lossless compression, while the JPEG is lossy.


%%%%%%%%%%%%%%%%%%%%%%%%%%%
% MATHEMATICAL BACKGROUND %
%%%%%%%%%%%%%%%%%%%%%%%%%%%
\section{Mathematical Background}


%%%%%%%%%%%%%%%
% APPLICATION %
%%%%%%%%%%%%%%%
\section{Application}


%%%%%%%%%%%%%%
% CONCLUSION %
%%%%%%%%%%%%%%
\section{Conclusion}


%%%%%%%%%%%%%%
% REFERENCES %
%%%%%%%%%%%%%%
\section{References}

\end{document}
