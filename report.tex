\documentclass{article}
\usepackage[utf8]{inputenc}
\usepackage {mathtools, graphicx, amsfonts, amssymb, comment}
\usepackage{enumerate}
\usepackage{geometry}
\usepackage{subcaption}
\usepackage{float}

\geometry{margin=1.25in}

% http://tex.stackexchange.com/questions/65849/confusion-onehalfspacing-vs-spacing-vs-word-vs-the-world
\linespread{1.25}

\title{Math 414 Final Project}
\author{Matt Gaikema \\ Will Argueta}
\date{April 2016}

\begin{document}

\maketitle

%%%%%%%%%%%%%%%%
% INTRODUCTION %
%%%%%%%%%%%%%%%%
\section{Introduction}
% https://en.wikipedia.org/wiki/Wavelet_transform

In order to efficiently send and receive images, it is often useful to compress them, 
since much of the data may be irrelevant or redundant.
One of the many ways of accomplishing this is through wavelets.

There are two types of image compression: lossy and lossless.
\textbf{Lossless} compression means that every bit of information is recovered from the compressed data,
while \textbf{lossy} compression occurs when redundant information is eliminated.
The GIF is an example of lossless compression, while the JPEG is lossy.

The JPEG 2000 is one method of image compression which uses wavelets.
It was created in 2000 with the aim of replacing the original JPEG standard, 
which uses the Discrete Cosine Transform.
The file extension is \verb|.jp2|, compared with the typical JPEG extension of \verb|.jpg|.

There are a few advantages of the JPEG 2000 over the JPEG.
First, JPEG 2000 can be lossy or lossless, while JPEG compression is always lossy.
JPEG 2000 produces images of much higher quality.

Of course, despite the advantages of the JPEG 2000, it is rarely used today, 
while the JPEG is still very popular.
The main reason is computation. 
Adopting the JPEG 2000 would have required rewriting much software, 
since it was not backwards-compatable with the JPEG.
Many companies were unwilling to add support for it since it wasn't popular,
and many consumers didn't use it since there was little support for it,
creating an unfortunate cycle. 
To this day, few websites and no major web browsers support it.
Additionally, the JPEG 2000 requires more computing power, which,
in 2000, was not as abundant as it is today.


%%%%%%%%%%%%%%%%%%%%%%%%%%%
% MATHEMATICAL BACKGROUND %
%%%%%%%%%%%%%%%%%%%%%%%%%%%
\section{Mathematical Background}
% https://en.wikipedia.org/wiki/JPEG_2000
% http://cs.haifa.ac.il/hagit/courses/seminars/wavelets/Presentations/Lecture09_Denoising.pdf

The wavelet transform on an image produces as many coefficients as there are pixels.


%%%%%%%%%%%%%%%
% APPLICATION %
%%%%%%%%%%%%%%%
\section{Application}


%%%%%%%%%%%%%%
% CONCLUSION %
%%%%%%%%%%%%%%
\section{Conclusion}


%%%%%%%%%%%%%%
% REFERENCES %
%%%%%%%%%%%%%%
\section{References}


\end{document}
