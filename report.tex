\documentclass{article}
\usepackage[utf8]{inputenc}
\usepackage {mathtools, graphicx, amsfonts, amssymb, comment}
\usepackage{enumerate}
\usepackage{geometry}
\usepackage{subcaption}
\usepackage{float}
\usepackage[backend=bibtex,sorting=ynt]{biblatex}

\addbibresource{bibliography.bib}
\geometry{margin=1.25in}

% http://tex.stackexchange.com/questions/65849/confusion-onehalfspacing-vs-spacing-vs-word-vs-the-world
\linespread{1.25}

\title{Math 414 Final Project}
\author{Matt Gaikema \\ Will Argueta}
\date{April 2016}

\begin{document}

\maketitle

%%%%%%%%%%%%%%%%
% INTRODUCTION %
%%%%%%%%%%%%%%%%
\section{Introduction}

In order to efficiently send and receive images, it is often useful to compress them, 
since much of the data may be irrelevant or redundant.
One of the many ways of accomplishing this is through wavelets.

There are two types of image compression: lossy and lossless.
\textbf{Lossless} compression means that every bit of information is recovered from the compressed data,
while \textbf{lossy} compression occurs when redundant information is eliminated.
The GIF is an example of lossless compression, while the JPEG is lossy.

The JPEG 2000 is one method of image compression which uses wavelets.
It was created in 2000 with the aim of replacing the original JPEG standard, 
which uses the Discrete Cosine Transform.
The file extension is \verb|.jp2|, compared with the typical JPEG extension of \verb|.jpg|.

% http://www.verypdf.com/pdfinfoeditor/jpeg-jpeg-2000-comparison.htm
\begin{figure}
	\centering
	\includegraphics[scale=0.4]{resources/comparison.png}
	\caption{Two images compressed using JPEG and JPEG 2000.\cite{comparison}}
	\label{fig:compare}
\end{figure}

There are a few advantages of the JPEG 2000 over the JPEG.
First, JPEG 2000 can be lossy or lossless, while JPEG compression is always lossy.
This means that JPEG 2000 produces images of much higher quality.
Figure \ref{fig:compare} shows two images compressed using JPEG and JPEG 2000.

Of course, despite the advantages of the JPEG 2000, it is rarely used today, 
while the JPEG is still very popular.
The main reason is computation.\cite{alternative} 
Adopting the JPEG 2000 would have required rewriting much software, 
since it was not backwards-compatable with the JPEG.
Many companies were unwilling to add support for it since it wasn't popular,
and many consumers didn't use it since there was little support for it,
creating an unfortunate cycle. 
Additionally, the JPEG 2000 requires more computing power, which was not as abundant in 2000 as it is today.
To this day, few websites and no major web browsers support it.


%%%%%%%%%%%%%%%%%%%%%%%%%%%
% MATHEMATICAL BACKGROUND %
%%%%%%%%%%%%%%%%%%%%%%%%%%%
\section{Mathematical Background}
% https://en.wikipedia.org/wiki/JPEG_2000

The JPEG 2000's use of the Discrete Wavelet Transform is what makes it superior to the JPEG.\cite{how}
The Discrete Cosine Transform (DCT), which is used by the JPEG, compresses the image into 8x8 blocks
and places them consecutively in the file.
The blocks are compressed individually.
This is the reason for "blockiness" in compressed JPEG images.

Wavelet compression converts the image into a series of wavelets, which can be more easily stored than pixel blocks.

The JPEG 2000 compression algorithm consists of four basic steps: 
preprocess, transformation, quantization, and encoding\cite{whydomath}.

The wavelet transform on an image produces as many coefficients as there are pixels.


%%%%%%%%%%%%%%%
% APPLICATION %
%%%%%%%%%%%%%%%
\section{Application}


%%%%%%%%%%%%%%
% CONCLUSION %
%%%%%%%%%%%%%%
\section{Conclusion}

The JPEG 2000 could easily be the next standard for image compression, if only people would use it.

%%%%%%%%%%%%%%
% REFERENCES %
%%%%%%%%%%%%%%
\printbibliography[]


\end{document}
