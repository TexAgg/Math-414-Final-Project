\documentclass{article}
\usepackage[utf8]{inputenc}
\usepackage {mathtools, graphicx, amsfonts, amssymb, comment}
\usepackage{enumerate}
\usepackage{geometry}

\geometry{margin=1.25in}

% http://tex.stackexchange.com/questions/65849/confusion-onehalfspacing-vs-spacing-vs-word-vs-the-world
\linespread{1.25}

\title{Math 414 Final Project}
\author{Matt Gaikema \\ Will Argueta}
\date{April 2016}

\begin{document}

\maketitle

%%%%%%%%%%%%%%%%
% INTRODUCTION %
%%%%%%%%%%%%%%%%
\section{Introduction}
% https://en.wikipedia.org/wiki/Wavelet_transform
% https://www.wolfram.com/mathematica/new-in-8/wavelet-analysis/

In order to efficiently send and receive images, it is often useful to compress them.
One of the many ways of accomplishing this is through the JPEG 2000 standard, which uses wavelet transforms.

There are two types of image compression: lossy and lossless.
\textbf{Lossless} compression means that every bit of information is recovered from the compressed data,
while \textbf{lossy} compression occurs when redundant information is eliminated.
The GIF is an example of lossless compression, while the JPEG is lossy.

A particular use for wavelets is in the JPEG 2000, which is an image compression standard.

Another spot where wavelets shine in image analysis is in denoising an image.

%%%%%%%%%%%%%%%%%%%%%%%%%%%
% MATHEMATICAL BACKGROUND %
%%%%%%%%%%%%%%%%%%%%%%%%%%%
\section{Mathematical Background}
% https://en.wikipedia.org/wiki/JPEG_2000

The wavelet transform on an image produces as many coefficients as there are pixels.


%%%%%%%%%%%%%%%
% APPLICATION %
%%%%%%%%%%%%%%%
\section{Application}


%%%%%%%%%%%%%%
% CONCLUSION %
%%%%%%%%%%%%%%
\section{Conclusion}


%%%%%%%%%%%%%%
% REFERENCES %
%%%%%%%%%%%%%%
\section{References}

\end{document}
